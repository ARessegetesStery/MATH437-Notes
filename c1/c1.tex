\documentclass{article}
\usepackage{../refalg}

\begin{document}
\Makepagesectionhead{MATH 437}{Manifolds}{ARessegetes Stery}

\tableofcontents
\newpage

\def\open{\mathcal{O}}

\section{Preliminary Results from Calculus}

This section is dedicated to proving the following two theorems, which has various implications in differential geometry:

\begin{theorem}[Inverse Function Theorem]\label{thm:inverse function theorem}
    Let $X$, $Y$ be finite-dimensional normed vector space, $\Omega \subseteq X$ an open set, and $f: \Omega \to Y$ a continuously differentiable function. Let $x_0 \in \Omega$. If $\restr{Df}{x_0}$ is invertible, then $f$ is locally $C^{1}$-differentiable at $x_0$ (i.e. $f^{-1}$ is $C^1$-continuous), whose derivative satisfies
    \[
        \restr{Df^{-1}}{f(x)} = (\restr{Df}{x})^{-1}
    \]
\end{theorem}

\begin{theorem}[Implicit Function Theorem]\label{thm:implicit function theorem}
    Let $\Omega \subseteq \R^n \times R^m$ be open, with $F \in C^1{\Omega, \R^m}$ s.t. $F(x_0, y_0) = 0$ for $(x_0, y_0) \in \Omega$. If
    \[
        \det \restr{DF(x_0, -)}{y_0} \neq 0
    \]
    then there exists $\varepsilon \in \R$, and $g \in C^1(B_{\varepsilon}(x_0), \R^m)$ s.t.
    \[
        F(x, g(x)) = 0 \qquad \qquad \forall \ x \in B_{\varepsilon}(x_0)
    \]
\end{theorem}

The general setup is to first prove the inverse function theorem, and use it to vastly simplify the proof for the implicit function theorem. An alternative approach is to first prove the implicit function theorem, and then express the derivative of the inverse function in an implicit manner. 

First setup some lemmas for further use in the proof:

\begin{lemma}[Contraction Mapping Principle]\label{lem: contraction mapping principle}
    Let $X$ be a complete normed vector space, and $M$ be a closed subset of $X$. If $f: M \to M$ is Lipschitz continuous with Lipschitz constant less than 1 (a contraction), then there exists $x_0 \in M$ s.t. $f(x_0) = x_0$.
\end{lemma}

\begin{proof}
    Since $X$ is complete and $M$ is closed, it suffices to prove that the sequence given by $(a_n) := f^n(x)$ is Cauchy. The limit must lie in $M$ as otherwise $M$ is not equal to its closure. By hypothesis there exists $C \in [0, 1)$ s.t.
    \[
        \abs{f^2(x) - f(x)} < C \cdot \abs{f(x) - x} \qquad \qquad \forall \ x \in M
    \]
    This gives the approximation
    \[
        \abs{f^m(x) - x} < \sum_{k=0}^{m-1} C^k \abs{f(x) - x} < \left( \sum_{k=0}^{\infty} C^k \right) \cdot \abs{f(x) - x} = \frac{1}{1 - C} \cdot \abs{f(x) - x}
    \]
    which implies that $f^m(x)$ is bounded for all $m$. Further for $m < n$ we have
    \[
        \abs{f^n(x) - f^m(x)} \leq \sum_{k=m}^{n-1} \abs{f^{k+1}(x) - f^k(x)} < \sum_{k=m}^{n-1} C^k \abs{f(x) - x} < C^m \left(\sum_{k=0}^{\infty} C^k \abs{f(x) - x}\right) = \frac{C^m}{1 - C}\abs{f(x) - x}
    \]
    which can be arbitrarily small via specifying $m$ to be arbitrarily large. Therefore $(a_n)$ admits a limit $a$. Then 
    \[
        a = \lim_{n \to \infty} a_n = \lim_{n \to \infty} f(a_n) = f(\lim_{n \to \infty} a_n) = f(a)
    \]
    since $f$ is continuous (as it is Lipschitz continuous), it commutes with taking the limit.
\end{proof}

\begin{lemma}
    Let $X$ be a normed vector space, with $L \in \{L \in \GL(X) \mid \norm{L} < 1\}$ then $(\Id - L) \in \GL(X)$. 
\end{lemma}

\begin{proof}
    Consider the series $S = \sum_{i=k}^{\infty} L^k$. This converges as
    \[
        \norm{S} \leq \sum_{i=k}^{\infty} \norm{L}^k = \frac{1}{1 - \norm{L}}
    \]
    which is finite and well-defined as $\norm{L} \in [0, 1)$. Then consider
    \[
        \lim_{n \to \infty}(\Id - L)\cdot S = \Id - \lim_{n \to \infty} L^n = \Id
    \]
    as $\norm{L} < 1$. 
\end{proof}

\section{Preliminary Topology}

\begin{definition}[Topological Space]
    A \textbf{topological space} is a pair (X, $\open$), where $X$ is an arbitrary set, and $\open$ a class of subsets of $X$, which satisfies the follows:
    \begin{itemize}
        \item $\emptyset \in \open$, and $X \in \open$.
        \item Let $(\Omega_i)_{i \in I}$ be a (possibly infinite) class of subsets of $X$, then $\bigcup_{i \in I} \Omega_i \in \open$.
        \item Let $(\omega_i)_{i \in I}$ be a finite class of subsets of $X$, then $\bigcap_{i \in I} \omega_i \in \open$.
    \end{itemize}
    Elements of $\open$ are called the \textbf{open sets}. $\open$ (as a class) gives the topology of $X$. 
\end{definition}

\begin{remark}
    If the topology of $X$ is clear, one often denotes the space by simply $X$. 
\end{remark}

\begin{definition}[Hausdorff Space]
    A topological space $(X, \open)$ is a \textbf{Hausdorff space} if for all $x, y \in X$ there exists $\Omega_x, \Omega_y \in \open$ s.t. $x \in \Omega_x, y \in \Omega_y$; and $\Omega_x \cap \Omega_y = \emptyset$.
\end{definition}

\begin{example}
    Choice of topology is very important. Consider the following examples:
    \begin{itemize}
        \item Consider $(\R, \open)$ where $\Omega \in \open$ if and only if for all $x \in \Omega$, there exists $\varepsilon \in \R$ s.t. $(x - \varepsilon, x + \varepsilon) \subseteq \Omega$. This space is Hausdorff as one could choose $\varepsilon_x = \varepsilon_y = \abs{x - y}/4$. This also illustrates why the intersection must be finite, as otherwise one could construct a converging sequence of $\varepsilon$, whose corresponding class of subsets intersecting to a closed interval.
        \item Consider $(\R, \open)$ to be the trivial topology, where $\open := \{\R, \emptyset\}$. Then the only subset in $\open$ containing $x$ and $y$ is $\R$, which implies that the space is not Hausdorff.
    \end{itemize}
\end{example}

\begin{definition}[Basis of a Topology]
    Let $(X, \open)$ be a topology space. Then a subset $\mathcal{B} \subseteq \open$ is a \textbf{basis of topology} $\open$ if for any $\Omega \in \open$ it can be expressed as union of elements in $\mathcal{B}$. If there exists such $\mathcal{B}$ s.t. it is countably infinite, then $X$ admits a \textbf{countable basis of topology}.
\end{definition}

\begin{proposition}
    $(\R, \open)$ with topology defined as in the first case in the example above admits a countable basis of topology.
\end{proposition}

\begin{proof}
    First consider $\mathcal{B} := \{(a, b) \mid a < b, a, b \in \R\}$. This gives a basis for the topology of $(\R, \open)$ in the sense of the first case above:
    \begin{itemize}
        \item \emph{Any union of elements in $\mathcal{B}$ is open.} Denote such union to be $\Omega$. By construction, for $x \in \R$ s.t. $x \in \Omega$, there exists $a_x < b_x$ s.t. $x \in (a_x, b_x)$. Then there exists $\varepsilon = \min\{x - a_x, b_x - x\} / 2$ that satisfies the definition.
        \item \emph{Any open subset of $\R$ can be expressed as a union of open intervals}. Recall that the union can be infinite. Now consider for $\Omega$ an open interval
        \[
            \Omega = \bigcup_{x \in \Omega} (x - \varepsilon_x, x + \varepsilon_x)
        \]
        where for each $x$, $\varepsilon_x$ is the corresponding radius s.t. the definition is satisfied; and for all $x \in \Omega$, $(\R \smallsetminus \Omega) \cap (x - 2\varepsilon_x, x + 2\varepsilon_x) \neq \emptyset$. By definition for all $x \in \Omega$ there exists such an interval that $x$ is in it.
    \end{itemize}
    Then consider $\mathcal{B}' := \{(a, b) \mid a < b, a, b \in \Q\}$. Since $\Q$ is dense in $\R$, for any $x' \in \R \cap \Omega$ there exists a sequence $(x_n)$ s.t. $\lim_{n \to \infty} x_n = x'$. Choose the sequence s.t. $\abs{x_n - x'} > 2\abs{x_{n+1} - x'}$. Suppose that for all $n$, $x' \notin (x_n - \varepsilon_{x_n}, x_n + \varepsilon_{x_n})$. Then there exists some $n_0$ s.t. $(x_{n+1} - \varepsilon_{x_{n+1}}, x_{n+1} + \varepsilon_{x_{n+1}}) \subsetneq (x_n, x')$ assuming $x_n < x'$ without loss of generality, which is a contradiction.

    Since $\Q \simeq \N \times \N$ which is countable, $\Q \times \Q$ is also countable, indicating that $\beta'$ gives a countable basis of topology on $\R$.
\end{proof}

\begin{remark}
    For $\R^n$, the \underline{standard topology} is defined as where a set $\Omega$ is open if and only if for every $x \in \Omega$ there exists $\varepsilon > 0$ s.t. $B_{\varepsilon}^n (x) \subseteq \Omega$. 
\end{remark}

\begin{corollary}
    By the same proof, and considering open intervals as $1$-balls on $\R^1$, this gives that $\R^n$ with the standard topology admits a countable basis of topology. 
\end{corollary}

\begin{definition}[Induced Topology]
    Let $(X, \open_X)$ be a topological space, and $Y \subseteq X$. Then there exists a definition for $\open_Y$ where $\Omega' \in \open_Y$ if and only if there exists $\Omega \in \open_X$ s.t. $\Omega' = \Omega \cap Y$. This is the \textbf{induced topology} on $Y$.
\end{definition}

\begin{definition}
    Let $(X_1, \open_1)$, $(X_2, \open_2)$ be topological spaces. Then
    \begin{itemize}
        \item A map $f: X_1 \to X_2$ is \textbf{continuous} if for all $\Omega \in \open_2$, $f^{-1}(\Omega) \in \open_1$. 
        \item A map $f: X_1 \to X_2$ is \textbf{homeomorphic} if it is invertible, and both $f$ and $f^{-1}$ are continuous. 
    \end{itemize}
\end{definition}

\begin{example}
    The map $f: [0, 2\pi) \to S^1, x \mapsto (\cos x, \sin x)$ is not homeomorphic, as for the arcs wrapped around the origin (e.g. $f([0, \pi/6) \cup (11\pi/6, 2\pi))$) the image is open, but the interval itself is not open.
\end{example}

\begin{definition}[Diffeomorphism]
    Let $\Omega_1, \Omega_2 \subseteq \R^n$ be open sets. Then a homeomorphism $f: \Omega_1 \to \Omega_2$ is a \textbf{diffeomorphism} if both $f$ and $f^{-1}$ are differentiable. Similarly one can consider $C^k$-diffeomorphism for $f$ and $f^{-1}$ being $k$-times differentiable.
\end{definition}

\begin{proposition}
    If $f$ is a diffeomorphism which is $n$-times differentiable, then $f^{-1}$ is also $n$-times differentiable.
\end{proposition}

\begin{proof}
    Proceed to prove this via induction on $k$:
    \begin{itemize}
        \item \emph{Base case}. For $k = 1$, this is by the definition of diffeomorphisms.
        \item \emph{Inductive step}. Suppose that this is proven for $k = m$. For $k = m+1 \leq n$, since $f$ is $C^n$-differentiable and locally injective everywhere, by inverse function theorem this gives
        \[
            g'(x) = (f'(g(x)))^{-1}
        \]
        where by hypothesis $f'$ and $g$ are both $C^m$-differentiable, and therefore so is their composition and $g'$.
    \end{itemize}
    This gives $g'$ is $C^{n-1}$-differentiable, which gives $g$ being $C^n$-differentiable.
\end{proof}

\begin{remark}
    A differentiable homeomorphism is not necessarily a diffeomorphism, as $f$ is not necessarily locally injective everywhere. Consider $f: x \mapsto x^3$ whose inverse is not differentiable at $0$. $f$ is not locally injective at 0.  
\end{remark}

\section{Differentiable Manifolds}

\end{document}
